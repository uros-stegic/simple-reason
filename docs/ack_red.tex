\documentclass[a4paper]{article}

\usepackage{color}
\usepackage{url}
%\usepackage[T2A]{fontenc} % enable Cyrillic fonts
\usepackage[utf8]{inputenc} % make weird characters work
\usepackage{graphicx}
\usepackage{caption}
\usepackage{subcaption}
\usepackage{verbatim}
\usepackage[english,serbian]{babel}
%\usepackage[english,serbianc]{babel} %ukljuciti babel sa ovim opcijama, umesto gornjim, ukoliko se koristi cirilica
\setcounter{tocdepth}{1}
\usepackage[unicode]{hyperref}
\hypersetup{colorlinks,citecolor=green,filecolor=green,linkcolor=blue,urlcolor=blue}
\usepackage{amsmath}
%\newtheorem{primer}{Пример}[section] %ćirilični primer
\newtheorem{primer}{Primer}[section]
\newtheorem{definicija}[primer]{Definicija}
\captionsetup[figure]{name=Grafik}
\graphicspath{{./images/}}

\begin{document}

\title{Svođenje problema ispitivanja zadovoljivosti baznih EUF formula na SAT Akermanovom redukcijom \\ \small{~\\Seminarski rad u okviru kursa\\Automatsko rezonovanje\\ Matematički fakultet}}

\author{
	Una Stanković, Uroš Stegić\\
	una\_stankovic@yahoo.com, mi10287@alas.matf.bg.ac.rs}
\date{18.~septembar 2017.}
\maketitle


\tableofcontents

\newpage


\section{Uvod}
\label{sec:uvod}

Projekat opisan u nastavku predstavlja seminarski rad iz predmeta Automatsko rezonovanje, na master studijama Informatike. Ovaj dokument je celina, zajedno uz k\^od javno dostupan na adresi: https://github.com/uros-stegic/simple-reason, i čini njegovu dokumentaciju.\\\\ 

U okviru ovog projekta, razvijeno je sintaksno okruženje za rad sa iskaznom logikom i logikom prvog reda. Definicija jezika će biti predstavljena u sekciji \ref{sec:uputstvo}. Pored samog jezika, razvijen je interpreter koji izvršava k\^od.\\\\ Obim seta alata potrebnih za primenu Akermanove redukcije se vremenom povećao, tako da ceo projekat ``simple reason'' sada predstavlja radni okvir koji podržava sintaksu iskazne, predikatske i jednakosne logike sa neinterpretiranim funkcijama, kao i osnovne transformacije nad sintaksičkim stablima.\\\\

U nastavku, biće predstavljena Akermanova redukcija, a potom dati uputstvo za korišćenje softvera, kao i kratak pregled dizajna i arhitekture.


\section{Akermanova redukcija}

U ovom delu ukratko će biti predstavljene teorijske osnove vezane za Akermanovu redukciju, koje se oslanjaju na poznavanje osnovnih pojmova iz Automatskog rezonovanja, kao što su: predikatska logika, negaciona normalna forma, funkcijski simboli i ostalo gradivo pokriveno kursom. \\\\ 
Određeni podskup predikatske logike se može svesti na SAT problem redukcijom polazne formule zapisane u jeziku prvog reda na ekvizadovoljivu formulu iskazne logike. Jedan način da se izvede ovakvo svođenje je postupak \textbf{Akermanove redukcije}.

Podskup logike prvog reda nad kojima je moguće izvršiti Akermanovu redukciju je:
$$ L = \{\neg, \land, \lor, =, \neq, f_i/\alpha_i, p_i/\beta_i\}$$

pri čemu su $f_i$ neitrerpretirani funkcijski simboli arnosti $\alpha_i$, $p_i$ predikatski simboli arnosti $\beta_i$. Simbol $=$ ($\neq$) se interpretira na uobičajen način: predikati se evaluiraju kao tačni ako su objekti jednaki (nejednaki). Važno je napomenuti da se Akermanova redukcija može primeniti isključivo nad formulama koje \textit{nisu kvantifikovane}.

\subsubsection*{Algoritam Akermanove redukcije}
\textbf{Ulaz:} Nekvantifikovana formula jednakosne logike sa neinterpretiranim funkcijskim simbolima.
\textbf{Izlaz:} Nekvantifikovana formula jednakosne logike bez neinterpretiranih funkcijskih simbola.
\textbf{Algoritam}
\begin{enumerate}
	\item Prevesti ulaznu formulu u negacionu normalnu formu
	\item Zameniti termove koji predstavljaju funkcijske simbole jedinstvenim identifikatorima (počev od unutrašnjih ka spoljašnjim)\\
	primer:\\
	$$f(f(x)) = 1 \lor f(x) \neq 2 \mapsto f_2 = 1 \lor f_1 \neq 2$$
	$$smene: f_1 = f(x), f_2 = f(f_1)$$
	\item za svaki funkcijski simbol iz smene, za svaki par argumenata dodati aksiomu funkcionalne konzistentnosi (\textit{functional consistency axiom}):
	$$(x = f_1 \implies f_2 = f_1) \implies f_2 = 1 \lor f_1 \neq 2$$
\end{enumerate}

\subsubsection*{Primer}
Ulaz: Formula F bez kvantifikatora u jednakosnol logici sa neinterpretiranim funkcijama:
$$x1 = x_2 \mapsto f (x_1) \neq f (x_2) \lor f (x_1) \neq f(x_3)$$
\begin{enumerate}
\item Transformisati F u negacionu normalnu formu 
\item Zameniti funkcijske termove jedinstvenim identifikatorima iznutra-ka-spolja\\

$$x_1 = x_2 \mapsto f_1 \neq f_2 \lor f_1 \neq f_3$$\\
Tada:\\
$$ f_1 = f (x_1),$$ 
$$f_2 = f (x_2),$$
$$f_3 = f (x_3)$$
\item Dodavanjem aksiome funkcionalne konzistentnosti za svaki par argumenata od f dobijamo:
$$(x_1 = x_2 \mapsto f_1 = f_2)$$
$$\land(x_1 = x_3 \mapsto f_1 = f_3)$$
$$\land(x_2 = x_3 \mapsto f_2 = f_3)$$
$$\mapsto (x_1 = x_2 \mapsto f_1 \neq f_2 \lor f_1 \neq f_3)$$
\end{enumerate}

\section{Uputstvo za korišćenje}
\label{sec:uputstvo}
Prilikom kreiranja projekta \textit{Simple reason}, omogućena su dva načina korišćenja programa: 
\begin{itemize}
	\item intreraktivni mod (REPL, read-evaluate-print-loop)
	\item izvršavanjem skript datoteke  
\end{itemize}

Oba načina biće predstavljena u nastavku, ali najpre će biti prikazana sintaksa jezika, kao i naredbe okruženja, kojima vršimo različite transformacije nad zadatom formulom.\\\\
Sintaksa jezika obuhvata:\\
\begin{tabular}{ |l|l| }
	\hline
	\multicolumn{2}{|c|}{Izrazi} \\
	\hline
	$\lnot F$ & $\sim F$ \\
	$F_1 \land F_2$ & $F1 \& F2$ \\
	$F_1 \lor F_2$ & $F1 | F2$ \\
	$F_1 \implies F_2$ & $F1 => F2$ \\
	$F_1 \Leftrightarrow F_2$ & $F1 <=> F2$ \\
	$\forall x. \Leftrightarrow F$ & $Ax. F$ \\
	$\exists x. \Leftrightarrow F$ & $Ex. F$ \\
	$pred. symbols$ & $[pqr][0-9]*$ \\
	$vars$ & $[xyz][0-9]*$ \\
	$UF$ & $[fgh][0-9]*$ \\
	$const$ & $[abc1-9][0-9]*$ \\
	\hline
\end{tabular}
\\\\
Naredbe okruženja:
\begin{itemize}
	\item \textbf{cnf F} - vraća kopiju formule F koja je u KNF
	\item \textbf{nnf F} - vraća kopiju formule F koja je u NNF
	\item \textbf{c\_distr F} - vraća kopiju formule F kojoj su konjunkcije spuštene u AST-u.
	\item \textbf{d\_distr F} - vraća kopiju formule F kojoj su disjunkcije spuštene u AST-u.
	\item \textbf{n\_distr F} - vraća kopiju formule F kojoj su negacije spuštene do listova AST-a.
	\item \textbf{simplify F} - vraća kopiju formule F koja je pojednostavljena\footnote{više o ovim transformacijama na: http://poincare.matf.bg.ac.rs/~filip/ar/ar-iskazna-logika.pdf}.
	\item \textbf{no\_imps F} - vraća kopiju formule F iz koje su eliminisane implikacije i ekvivalencije
	\item \textbf{quant\_up F} - vraća kopiju formule F u kojoj su izdignuti svi kvantifikatori do korena AST-a.
	\item \textbf{prenex F} - vraća kopiju formule F koja je u preneks normalnoj formi.
	\item \textbf{ack F} - primenjuje Akermanovu redukciju nad kopijom formule F.
	\item \textbf{\$a := F} - deklariše novu meta promenljivu u okruženju.
	\item \textbf{delete \$a} - uklanja meta promenljivu iz okruženja.
\end{itemize}

\subsubsection{Primer upotrebe}
\subparagraph{Interaktivni mod}
U ovom delu će biti predstavljen primer korišćenja \textit{Simple reason} softvera u interaktivnom modu:
\begin{verbatim}
> $a := Ax.Ey.p(x) => q(y)
> $a
Ax.Ey. p(x) => q(y)
> nnf $a
Ax.Ey.~p(x) | q(y)
> quit

Quiting.
\end{verbatim}

\subparagraph{Skript mod}
Osim interaktivnog omogućen je i skript mod, najpre u svrhu mogućnosti jednostavnog izvođenja testova, kao i da bi se korisniku omogućilo da na lakši način koristi program.\\ 
Osnovna ideja je da korisnik piše skript u datoteci koji kasnije prosleđuje programu kao argument komandne linije. Pogledajmo primer ovakve upotrebe:

\bigskip
Datoteka ``primer1.fml''
\begin{verbatim}
$a := p(x)
$b := q(x)
nnf $a => $b
\end{verbatim}


\bigskip
Pokretanje iz komandne linije
\begin{verbatim}
[user@pc code]$ ./simple-reason primer1.fml
Ax.Ey.~p(x) | q(y)

Quiting.
\end{verbatim}

\section{Pregled dizajna i arhitekture}
\label{sec:dizajn}
Dizajn softvera je analogan onome koji je rađen na vežbama. Razlikuje se po tome što su sve vrednosti deklarisane kao konstantne kako bismo se približili funkcionalnom aspektu jezika C++.\\
Sve transformacije nad formulama su izdvojene u zasebne celine u maniru uzorka projektovanja ``posetilac''. Za posetioca je odlučeno zato što omogućava definisanje nove operacije bez izmene klasa elemenata nad kojima se izvršava. Formule koje se posećuju su imutabilne strukture podataka i svaka transformacija formule vraća novo sintaksno stablo.

\subsection{Organizacija projekta}
Projekat je organizovan u nekoliko foldera:
\begin{enumerate}
\item include - koji sadrži .hpp fajlove i sastoji se od podfoldera:
	\begin{itemize}
	\item formulae - u kome su deklarisani atomička formula, formula, binarne i unarna operacija, konstante, promenljive i kvantifikatori, itd.,
	\item transformations - u kome su deklarisane moguće transformacije nad elementima formule, kao što su: izvlačenje kvantifikatora, distribucija konjunkcije, negacije, prebacivanje u KNF, NNF, PRENEX, akermanova redukcija, itd.,
	\end{itemize}
\item src - koji sadrži .cpp fajlove i sastoji se od:
	\begin{itemize}
	\item podfoldera formulae,
	\item podfoldera transformations - u kojima su definicije funkcija,
	\item fajla main.cpp,
	\end{itemize}
\item parser - u kome se nalaze fajlovi parser.l i lexer.ypp,  
\item tests - u kome se nalaze .fml fajlovi koji služe za testiranje projekta pri pokretanju iz komandne linije,
\item docs - u kome se nalazi projektna dokumentacija.\\\\
Za ovakvu organizaciju strukture fajlova smo se odlučili jer nam nudi najveću preglednost i jednostavnost pri dodavanju novih fajlova i doprinosi lakšem razumevanju.
\end{enumerate}

\section{Zaključak}

Najveći doprinos ovog rada predstavlja upravo program, koji na jasan način prikazuje šta se dobija primenom Akermanove redukcije na polaznu formulu. Osim toga, k\^od može biti korišćen i za prikaz raznih drugih transformacija nad polaznom formulom (svođenje na PRENEX, NNF,itd.), što zainteresovanom korisniku može biti korisno.

\end{document}